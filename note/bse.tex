\documentclass[disablejfam,12pt]{article}
\usepackage{graphicx}
\usepackage{color}
\pagestyle{empty}
\setlength{\textheight}{240mm}
\setlength{\textwidth}{160mm}
\setlength{\columnsep}{5mm}
\setlength{\topmargin}{-30mm}
\setlength{\oddsidemargin}{0mm}
\setlength{\evensidemargin}{0mm}
\usepackage{amssymb}
\usepackage{amsmath}
\usepackage{braket}

\newcommand{\mathi}{\ensuremath{\mathrm{i}}}
\newcommand{\mi}{\ensuremath{\mathrm{i}}}
\newcommand{\barT}{\ensuremath{\bar{T}}}

\author{Hiroshi SHINAOKA}

\begin{document}
\title{BSE}
\maketitle
\thispagestyle{empty}



\section{Two-particle Green's function}

\subsection{ALPS/CT-HYB}
\begin{eqnarray}
 G_{pqrs}(\tau_1, \tau_2, \tau_3, \tau_4) &\equiv& \braket{T_\tau c_p(\tau_1) c_q^\dagger(\tau_2) c_r(\tau_3) c_s^\dagger(\tau_4)}\\
 G_{pqrs}(i \omega_n, i \omega_{n^\prime}, i\nu_m) &=& \int_0^\beta  d \tau_{12} d \tau_{34} d \tau_{14} G_{pqrs}(\tau_{12}, \tau_{34}, \tau_{14}, 0) e^{i\omega_n \tau_{12}} e^{i\omega_{n^\prime} \tau_{34}}e^{i \nu_m \tau_{14}}
\end{eqnarray}

\subsection{Otsuki's BSE}

\begin{align}
X^\mathrm{loc}_{pqrs}(\tau_1, \tau_2, \tau_3, \tau_4) &\equiv \langle T c^\dagger_p(\tau_1) c_q(\tau_2) c^\dagger_r(\tau_3) c_s(\tau_4)\rangle\nonumber\\
X^\mathrm{loc}_{pqrs}(i \omega_n, i \omega_{n^\prime}, i\nu_m) &= T \int_0^\beta  d \tau_{12} d \tau_{34} d \tau_{14} X^\mathrm{loc}_{pqrs}(\tau_{12}, \tau_{34}, \tau_{14}, 0) e^{i\omega_n \tau_{12}} e^{i\omega_{n^\prime} \tau_{34}}e^{i \nu_m \tau_{14}}
\end{align}

Now, we relate the two notations.
\begin{align}
G_{pqrs}(\tau_{12}, \tau_{34}, \tau_{14}, 0) &= X^\mathrm{loc}_{srqp}(-\tau_{34}, -\tau_{12}, -\tau_{14, 0})\\
X^\mathrm{loc}_{pqrs}(\mi \omega_n, \mi \omega_{n^\prime}, \mi \nu_m) &= T G_{srqp}(-\mi \omega_{n^\prime}, -\mi \omega_n, -\mi \nu_m) = T G_{srqp}(\mi \omega_{-n^\prime-1}, \mi \omega_{-n-1}, \mi \nu_{-m}).
\end{align}

\end{document}
